% IEEEAerospace2012.cls requires the following packages: times, rawfonts, oldfont, geometry
\documentclass[twocolumn,letterpaper]{IEEEAerospaceCLS}  % only supports two-column, letterpaper format

% The next line gives some packages you may find useful for your paper--these are not required though.
%\usepackage[]{graphicx,float,latexsym,amssymb,amsfonts,amsmath,amstext,times,psfig}
% NOTE: The .cls file is now compatible with amsmath!!!

\usepackage[]{graphicx}    % We use this package in this document
\newcommand{\ignore}[1]{}  % {} empty inside = %% comment

\begin{document}
\title{Planetary Rover Simulation for Lunar Exploration Missions}

\author{%
}

\maketitle

\thispagestyle{plain}
\pagestyle{plain}

\maketitle

\thispagestyle{plain}
\pagestyle{plain}

\begin{abstract}

When planning planetary rover missions it is useful to develop intuition and skills driving in, quite literally, alien environments before incurring the cost of reaching said locales. Simulators make it possible to operate in environments that have the physical characteristics of target locations without the expense and overhead of extensive physical tests. To that end, NASA Ames and Open Robotics collaborated on a lunar rover driving simulator based on the open source Gazebo simulation platform and leveraging ROS (Robotic Operating System) components. The simulator was integrated with research and mission software for rover driving, system monitoring, and science instrument simulation to constitute an end-to-end lunar mission simulation capability.

Although we expect our simulator to be applicable to arbitrary lunar regions, we designed to a reference mission of prospecting in polar regions. The harsh lighting and low illumination angles at the lunar poles combine with the unique reflectance properties of lunar regolith to present a challenging visual environment for both human and computer perception. Our simulator placed an emphasis on high fidelity visual simulation in order to produce synthetic imagery suitable for evaluating human rover drivers with navigation tasks, as well as providing test data for computer vision software development.

In this paper, we describe the software used to construct the simulated lunar environment and the components of the driving simulation. Our synthetic terrain generation software artificially increases the resolution of lunar digital elevation maps by fractal synthesis and inserts craters and rocks based on lunar size-frequency distribution models. We describe the necessary enhancements to import large scale, high resolution terrains into Gazebo, as well as our approach to modeling the visual environment of the lunar surface. An overview of the mission software system is provided, along with how ROS was used to emulate flight software components that had not been developed yet.

We summarize how the simulator has been used to refine the mission concept of operations and to evaluate the operations impact of rover engineering design decisions. We reduced uncertainty about mission operations tempo by simulating mission scenarios with representative drive speeds, telemetry rates and network delays. The operations team exploited the simulator’s flexibility to experiment with different rover configurations and compare the effect of things such as different camera placement options and mobility system steering constraints.

Finally, we discuss the effect of using the high-fidelity synthetic lunar images for visual odometry.  We also discuss the characterization of the shader model for lunar illumination relative to a ray-tracing with accurate reflectance models.  Further, we characterize the wheel slip model, and find some inconsistencies in the produced wheel slip behaviour.

\end{abstract}


\tableofcontents

%%%%%%%%%%%%%%%%%%%%%%%%%%%%%%%%%%%%%%
\section{Introduction}
%%%%%%%%%%%%%%%%%%%%%%%%%%%%%%%%%%%%%%


- Resource Prospector Background
  - High level overview of mission objectives
  - Differences to Mars ops
  - High level concept of operations
  - High level software architecture
    - flight/ground software split
- Simulator goals / overview
  - Development approach
    - create driving sim as fast as possible
      - development/refinement of conops
    - prototype mission software architecture with stubs and existing components
    - use sim-generated data for flight software development
      - flight software --> Gazebo
      - Gazebo and ROS output --> nav and localization
    - incrementatlly replace stubs and emulated components with flight versions
    - use for training
- Synthetic Terrain Generation
- Lunar Visual Environment
  - Gazebo terrain enhancements
    - LOD
    - backround tiles
  - Shadows
    - real time shadow challenges 
    - baked shadows
    - Gazebo shadow improvements
  - GLSL shader
  - Wheel tracks
  - Ephemeris
  - HDR
    - Earthshine, bounced light
    - high bit depth rendering in Gazebo
    - PBR indirect lighting experiment 
    - attempt to reproduce in GLSL 
- Vehicle simulation
  - Use existing, similar robots to emulate 
    - Husky
    - KRex2
  - Flight software integration with Gazebo
  - Wheel slip plugin
    - first order approximation
    - fault injection
- Flight Software Prototype
  - flight/ground split
  - emulation with ROS components
  - Ops software
    - WARP
    - VERVE
  - Science sim and software
- Driving ConOps Experiments
- Results
  - visual odometry
  - lunar illumination comparison 
  - wheel slip results
  


  
  
Synthetic Terrain Generation

To produce a viable lunar driving experience the simulated world must be representative of the target physical environment. Specifically, the morphology of the terrain should accuarately represent terrain features that would be considered either positive or negative obstacles for the rover. Although DEMs are available for much of the lunar surface, the resolution of these models are not sufficient to represent rover-scale hazards. To make matters worse, the best resolution lunar DEMs are generated from stereo orbital imagery, yet the reference mission was targeting areas with permanently shadowed regions. 

Because the reference mission defined rocks larger than 10-25cm as positive obstacles, we desired a terrain resolution of 5cm or greater. Given that 5cm resolution is well beyond available lunar DEMs, we developed tools to artificially increase resolution using established techniques {parkes2004planet}. 



  
  
\end{document}
