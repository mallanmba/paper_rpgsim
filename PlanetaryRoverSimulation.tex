% IEEEAerospace2012.cls requires the following packages: times, rawfonts, oldfont, geometry
\documentclass[twocolumn,letterpaper]{IEEEAerospaceCLS}  % only supports two-column, letterpaper format

% The next line gives some packages you may find useful for your paper--these are not required though.
%\usepackage[]{graphicx,float,latexsym,amssymb,amsfonts,amsmath,amstext,times,psfig}
\usepackage[]{graphicx,float,latexsym,amssymb,amsfonts,amsmath,amstext,times}
% NOTE: The .cls file is now compatible with amsmath!!!

\usepackage[ampersand]{easylist}
\usepackage[]{graphicx}    % We use this package in this document
\newcommand{\ignore}[1]{}  % {} empty inside = %% comment

\begin{document}
\title{Planetary Rover Simulation for Lunar Exploration Missions}

\author{%
}

\maketitle

\thispagestyle{plain}
\pagestyle{plain}

\maketitle

\thispagestyle{plain}
\pagestyle{plain}

\begin{abstract}

When planning planetary rover missions it is useful to develop intuition and skills driving in, quite literally, alien environments before incurring the cost of reaching said locales. Simulators make it possible to operate in environments that have the physical characteristics of target locations without the expense and overhead of extensive physical tests. To that end, NASA Ames and Open Robotics collaborated on a lunar rover driving simulator based on the open source Gazebo simulation platform and leveraging ROS (Robotic Operating System) components. The simulator was integrated with research and mission software for rover driving, system monitoring, and science instrument simulation to constitute an end-to-end lunar mission simulation capability.

Although we expect our simulator to be applicable to arbitrary lunar regions, we designed to a reference mission of prospecting in polar regions. The harsh lighting and low illumination angles at the lunar poles combine with the unique reflectance properties of lunar regolith to present a challenging visual environment for both human and computer perception. Our simulator placed an emphasis on high fidelity visual simulation in order to produce synthetic imagery suitable for evaluating human rover drivers with navigation tasks, as well as providing test data for computer vision software development.

In this paper, we describe the software used to construct the simulated lunar environment and the components of the driving simulation. Our synthetic terrain generation software artificially increases the resolution of lunar digital elevation maps by fractal synthesis and inserts craters and rocks based on lunar size-frequency distribution models. We describe the necessary enhancements to import large scale, high resolution terrains into Gazebo, as well as our approach to modeling the visual environment of the lunar surface. An overview of the mission software system is provided, along with how ROS was used to emulate flight software components that had not been developed yet.

We summarize how the simulator has been used to refine the mission concept of operations and to evaluate the operations impact of rover engineering design decisions. We reduced uncertainty about mission operations tempo by simulating mission scenarios with representative drive speeds, telemetry rates and network delays. The operations team exploited the simulator’s flexibility to experiment with different rover configurations and compare the effect of things such as different camera placement options and mobility system steering constraints.

Finally, we discuss the effect of using the high-fidelity synthetic lunar images for visual odometry.  We also discuss the characterization of the shader model for lunar illumination relative to a ray-tracing with accurate reflectance models.  Further, we characterize the wheel slip model, and find some inconsistencies in the produced wheel slip behaviour.

\end{abstract}


\tableofcontents

%%%%%%%%%%%%%%%%%%%%%%%%%%%%%%%%%%%%%%
\section{Introduction}
%%%%%%%%%%%%%%%%%%%%%%%%%%%%%%%%%%%%%%

\begin{easylist}
& Main item~:
&& Sub item.
&& Another sub item.
\end{easylist}

\begin{easylist}
& Resource Prospector Background
&& High level overview of mission objectives
&& Differences to Mars ops
&& High level concept of operations
&& High level software architecture
&&& flight/ground software split
& Simulator goals / overview
  && Development approach
    &&& create driving sim as fast as possible
    &&& development/refinement of conops
  && prototype mission software architecture with stubs and existing components
  && use sim-generated data for flight software development
      &&& flight software --> Gazebo
      &&& Gazebo and ROS output --> nav and localization
    && incrementatlly replace stubs and emulated components with flight versions
    && use for training
& Synthetic Terrain Generation
& Lunar Visual Environment
  && Gazebo terrain enhancements
    &&& LOD
    &&& backround tiles
  && Shadows
    &&& real time shadow challenges 
    &&& baked shadows
    &&& Gazebo shadow improvements
  && GLSL shader
  && Wheel tracks
  && Ephemeris
  && HDR
    &&& Earthshine, bounced light
    &&& high bit depth rendering in Gazebo
    &&& PBR indirect lighting experiment 
    &&& attempt to reproduce in GLSL 
& Vehicle simulation
  && Use existing, similar robots to emulate 
    &&& Husky
    &&& KRex2
  && Flight software integration with Gazebo
  && Wheel slip plugin
    &&& first order approximation
    &&& fault injection
& Flight Software Prototype
  && flight/ground split
  && emulation with ROS components
  && Ops software
    &&& WARP
    &&& VERVE
  && Science sim and software
& Driving ConOps Experiments
& Results
  && visual odometry
  && lunar illumination comparison 
  && wheel slip results
 \end{easylist} 


  
  
\section{Synthetic Terrain Generation}

To produce a viable lunar driving experience the simulated world must be representative of the target physical environment. Specifically, the morphology of the terrain should accuarately represent terrain features that would be considered either positive or negative obstacles for the rover. Although DEMs are available for much of the lunar surface, the resolution of these models are not sufficient to represent rover-scale hazards. To make matters worse, the best resolution lunar DEMs are generated from stereo orbital imagery, yet the reference mission was targeting areas with permanently shadowed regions. 

Because the reference mission defined rocks larger than 10-25cm as positive obstacles, we desired a terrain resolution of 5cm or greater. Given that 5cm resolution is well beyond available lunar DEMs, we developed tools to artificially increase resolution using established techniques {parkes2004planet}. ** insert summary of PANGU paper **

The diamond-and-square method of fractal terrain generation is straightforward to implement and a reasonable choice for adding higher frequency interest. However, the method is known to create interpolation artifacts when constrained {miller1986definition}, particularly on low frequency features like aged crater rims. We employed two approaches to mitigating these artifacts. First, we replace linear interpolation with Catmull-Rom splines in the diamond-and-square algorithm which reduces the number of artifacts. Second, pior to scaling up we split the terrain into high and low frequency components by running a low pass filter on the DEM, and subtracting it from the orginal. The high frequency component is scaled up with the diamond-and-square algorithm whereas the low frequency component is scaled up with bilinear interpolation. Craters are inserted into the low frequency component before adding the high and low frequency components back together to reconstitute the DEM. This process is repeated until the desired resolution is reached. 
  
\section{Lunar Visual Environment}
some text

\subsection{Shadows}

\section{Lunar Visual Environment}
some text

\subsection{Shadows}
Shadow rendering provides countless challenges and has been under continuous development since the advent of computer graphics. Like most aspects of real-time computer graphics, a good solution involves finding a set of approximations that work well for you particular problem. Our problem was to render shadows that would appear as realistic as possible from the perspective of our rover's cameras. And, since one of our simulation goals was exploration of polar regions, we needed a shadow solution that would support low illumination angles (e.g. the sun close to the horizon).

For our purposes, we were only concerned with shadows cast by the sun. On a rover, any lights would necessarily be mounted close to the rover's cameras, so the shadows they cast would be mostly hidden from view.

The dominant real-time shadow technique and the one we used is called shadow mapping. It involves rendering a texture map that contains the depth of objects in a scene from the point-of-view of a light source and then comparing these depths with object positions when rendering the scene from the user's point-of-view. These comparisons reveal which surfaces are illuminated by the light and which are in shadow. There are other shadow techniques, but shadow mapping provides a popular combination of speed and flexibility.

\subsubsection{Real-time Shadow Challenges}
One common challenge related to shadow mapping is providing an appropriate amount of shadow map resolution in every part of a scene. Ideally, shadow detail decreases with distance from the user's point of view such that there is a unique shadow map texel for every pixel in the final image. In a small scene (e.g. indoors) a uniform distribution of shadow resolution can work well enough. However, when rendering a landscape out to the horizon, a uniform distribution of shadow resolution is impractical from the standpoint of memory consumption.

A related challenge stems from the relationship of the light ray direction compared to the user's view direction. When these directions are close to parallel, as is often the case in our simulation because the sun is close to the horizon, it becomes difficult to make a good distribution of shadow resolution. This is known as the dueling frusta problem. There are refinements of shadow mapping that exaggerate this problem and others that minimize it.

It is possible to improve shadow quality if you know, for a given scene, which shadows are static and which are dynamic. If the light source moves then all the shadows it casts are dynamic. The orientation of the sun relative to the moon changes so slowly that we can assume it is stationary in our simulation. The only moving object is the rover itself. Therefore, we can pre-compute shadows cast by the terrain (often referred to as baked shadows), and we only need to compute real-time shadows for the rover.

\subsubsection{Baked Shadows}
Useful to cover large area, show diffuse shadows from distant shadow casters. EXPAND ON THIS.

\subsubsection{Gazebo Shadow Improvements}
Gazebo originally used a shadow mapping technique called Parallel Split Shadow Maps (PSSM) to render real-time shadows. On the surface, this was a good choice because PSSM uses multiple shadow maps to provide a quality distribution of shadow resolution, even in the dueling frusta case. However, there were several problems with the implementation that conspired to ruin the shadows in our simulation, often leaving us with low-resolution shadows or no shadow on parts of the screen.

Gazebo uses Ogre3D for rendering, and so it was using Ogre3D's PSSM implementation for rendering shadows. Each shadow map in that implementation was also being processed by a Light-Space Perspective Shadow Map (liSPSM) technique, which performs poorly in the dueling frusta case and also exhibited some mysterious bugs. Additionally, Ogre3D assumes all scenes are oriented with the y-axis pointing up while gazebo assumes the z-axis points up. This decreased the usable region of each shadow map and often caused shadow edges to be more jagged than necessary. We corrected both problems by adding code to Gazebo that bypasses these troublesome parts of Ogre3D.

To further refine the PSSM implementation, we added hooks to the Gazebo API that allow aspects of the shadows to be tuned through Gazebo plugins. We were then able to affect the relative size of the shadow maps and the distance from the observer within which shadows would be applied.

With the resolution distribution problems mitigated, we turned to the overall lack of shadow resolution. The easiest first step was to increase Gazebo's default shadow map resolution from 1024 to 2048. This double our resolution everywhere, but there were still small features on the rover, such as the mast, that were not casting shadows consistently. The shader that Gazebo was using to apply shadows was smoothing shadow edges using Percentage Closest Filtering (PCF) with a 3x3 grid of sample positions. This worked well for smoothing jagged shadow edges, but, due to the filter's symmetry, sometimes small features would fall through the cracks and not affect the final shadow intensity. We replaced PCF with something known as hardware PCF, a common graphics driver hack that accomplishes the same type of smoothing with little performance impact. Then we applied a 9-sample Poisson filter (not a uniform grid like before), which smoothed the shadows even more and did a much better job of preserving small features.


\end{document}
